\documentclass[12pt]{article}
\setlength{\oddsidemargin}{0in}
\setlength{\evensidemargin}{0in}
\setlength{\textwidth}{6.5in}
\setlength{\parindent}{0in}
% \setlength{\parskip}{\baselineskip}
\usepackage{graphicx}
\usepackage{multirow}
\usepackage{multicol}
\usepackage{listings}
\usepackage{color}
\rmfamily
\definecolor{dkgreen}{rgb}{0,0.6,0}
\definecolor{gray}{rgb}{0.5,0.5,0.5}
\definecolor{mauve}{rgb}{0.58,0,0.82}

\lstset{frame=tb,
  language=R,
  aboveskip=3mm,
  belowskip=3mm,
  showstringspaces=false,
  columns=flexible,
  basicstyle={\small\ttfamily},
  numbers=none,
  numberstyle=\tiny\color{gray},
  keywordstyle=\color{blue},
  commentstyle=\color{dkgreen},
  stringstyle=\color{mauve},
  breaklines=true,
  breakatwhitespace=true,
  tabsize=3
}

\usepackage{amsmath,amssymb,amsrefs}
\usepackage[top=24mm, bottom=18mm, left=15mm, right=13mm]{geometry}
\usepackage{url}

\begin{document}
1. Considering the Laplacian denoted by $L$, then $G$ can be shown to be $k$-connected components if there $L$ can be broken into $k$ block diagonal matrices, which we can denote as $L_1, L_2,\dots,L_k$. That is:
\[ L = 
\begin{bmatrix}
L_1 & 0 & \dots & 0 \\
0 & L_2 & \ddots & 0 \\
\vdots & \ddots & \ddots & 0 \\
0 & \dots & \dots & L_k \\
\end{bmatrix}
\]
This is true since if there are separate components, then there is no edge between these components, so adjacency between points of different components should be 0. \\
Now by the property of Laplacians, the row sum and column sum of L is zero, so for all $L_1, L_2, \dots, L_k$, their row sums and column sums is 0. Thus, each have $\lambda_0 = 0$ with eigenvector $v_0 = (1,1,\dots, 1)$. Since there are $k$ blocks, each with 1 eigenvalue 0, and since $det(L) = det(L_1)det(L_2)\dots det(L_k)$, then $L$ has k components which is the multiplicity of the eigenvalue 0. 
\newpage
2.
\begin{lstlisting}
EnglandTeams = read.csv("England_2009_2010_TeamNames.csv", header = FALSE);
EnglandTeamScores = read.csv("England_2009_2010_Scores.csv", header = FALSE);

n = dim(EnglandTeamScores)[1];
S_match = (matrix(20, nrow=n, ncol=n) + as.matrix(EnglandTeamScores) %*% as.matrix(t(EnglandTeamScores)))/2;

D = diag(rowSums(S_match));

Laplacian = D - S_match;

E = eigen(Laplacian);
Ranking = E$vectors[,n-1];
Ranked = data.frame(Ranking, row.names = EnglandTeams[,1]);
Ranked[order(-Ranked$Ranking), , drop = FALSE];
\end{lstlisting}
This gives us the following ranking:\\ \\
\begin{tabular}{|c|c|}
\hline
'Burnley'     & 0.477379966\\
'Hull'        & 0.374209712\\
'Wigan'       & 0.285469630\\
'Portsmouth'  & 0.183143343\\
'Wolves'      & 0.183123190\\
'Sunderland'  & 0.119239116\\
'West Ham'    & 0.091111123\\
'Birmingham'  & 0.017176251\\
'Bolton'      & 0.014844431\\
'Stoke'       & 0.010421174\\
'Fulham'      & 0.002628019\\
'Blackburn'   &-0.002718788\\
'Liverpool'   &-0.084034033\\
'Aston Villa' &-0.099229806\\
'Everton'     &-0.157429641\\
'Arsenal'     &-0.174783324\\
'Tottenham'   &-0.273562003\\
'Man United'  &-0.291213755\\
'Chelsea'     &-0.329805720\\
\hline
\end{tabular}
\end{document}
